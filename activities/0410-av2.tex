\documentclass[a4paper, 11pt]{article}

\usepackage{fullpage}
\usepackage{hyperref}
\usepackage{amsthm}
\usepackage[numbers,sort&compress]{natbib}
\usepackage{listings}

\theoremstyle{definition}
\newtheorem{exercise}{Exercise}

\begin{document}
%%% Header starts
\noindent{\large\textbf{IS-521 Activity}\hfill
                \textbf{Anti-anti-malware} \\
         {\phantom{} \hfill Due Date: April 17, 2017 (before class)} \\
%%% Header ends

\section{Activity Overview}

In this activity, you will improve upon the malware you designed
(myworm) in the previous activity. The major goal in this activity is
to apply several anti-AV techniques we learned in the class to your
worm.

You should use the existing worm-* private repo. Specifically, you
should create a ``2nd-activity'' branch forked off from your own
master branch, and push all your modifications to the branch. You
\emph{must} fork your 2nd-activity branch from the \emph{head} of your
own master branch. In other words, you should use the latest code from
your master. Your final submission should be a pull request from the
branch to your own master.

You can modify your worm (at master branch) before you start this
activity if your worm was not complete at the moment of prior
submission. In any case, you should clearly fork off from the head of
your master branch.

\section{Part 1}

One of the simplest anti-forensic techniques we learned in the class
is to remove the running binary itself. This makes it difficult for
defenders to collect malware samples. Furthermore, the AV we designed
in the previous activity would not detect your worm from the
process-scanning mode, because the corresponding binary file simply
does not exist.

To achieve this, you can simply call \texttt{unlink} function as soon
as your worm starts running. You can further confuse the AV by
modifying the process name as well: the proc file system will reveal a
different path other than \texttt{myworm} when your worm is running.

However, this technique introduces a small problem: your worm needs to
propagate, but you do not have the program binary anymore on your
disk. Thus, it is desirable to keep the ability of the worm while
making it undetectable. To summarize,
%
\begin{enumerate}

  \item Your worm needs to run in the same way as in the previous
    activity except that it does some anti-AV tricks at the beginning.

  \item Specifically, your worm deletes itself when it starts.

  \item It also changes the name of the process, so the process
    appears to have a different name other than \texttt{myworm} when
    we see the running processes using \texttt{ps} command. Be
    creative. Which name would you use to confuse users? Hint:
    ``\texttt{man prctl}''.

  \item Even though the binary of your worm does not present in your
    disk, your worm should be able to propagate through the network by
    exploiting the vulnet vulnerability.

\end{enumerate}

\section{Part 2}

In this part, we consider an improved version of vulnet
server~\cite{vulnet}. You should be able to find the version in the
``dumb-protection'' branch of the vulnet repository. The improved
version now performs a series of simple checks in order to prevent the
authenticated superuser from obtaining the user name and the password
of vulnet processes. However, this protection mechanism has some
vulnerabilities. Re-design your worm to bypass the defense and get the
user name and the password of the running processes. Your worm should
be able to exploit both the previous and the modified version of
vulnet. Notice there are \emph{multiple} ways to exploit this
vulnerability.

Hint: One easy way to attack the new protection mechanism is to use a
bash trick: you can dynamically generate command line strings with the
help of several bash expressions. See the ``EXPANSION'' section of the
man page of bash: \texttt{man bash}.

\section{Deliverables}

We expect you to push your source code to your own private repository
(worm-your\_id). Your final submission for this activity is a pull
request to your own master branch from your 2nd-activity branch.

You \emph{must} follow the file naming convention described below. You
are allowed to add extra files that are not specifically mentioned in
the list below such as 3rd-party libraries or configuration files, but
you should always include the followings.  \textbf{If otherwise,
course staffs will immediately deduct half of the available score}.

\begin{enumerate}

  \item \textbf{./Makefile} file. We should be able to build your worm
    simply by typing ``make''. We will use the VM that we provided in
    the course to compile your worm. If there is any necessary
    dependencies to build your worm, your Makefile \emph{must}
    automatically install them. The produced binary should be placed
    in \textbf{./build/myworm}.

  \item \textbf{./src/} directory contains the source of your worm.
    You can use any language you want. N.B. Points will be deducted
    for the lack of necessary comments. Please make your code
    readable.

  \item \textbf{./README.md} file contains a write-up. This write-up
    should have three sections. The first section is ``Deliverables'',
    which simply lists what you did for each directory and file. The
    second section is ``Summary'', which essentially summarizes how
    you solved the problems in Part 1 and 2. Finally, the last section
    is ``Lesson''. In this section, you write what you learned from
    this activity.

\end{enumerate}

\bibliography{references}
\bibliographystyle{plainnat}

\end{document}
